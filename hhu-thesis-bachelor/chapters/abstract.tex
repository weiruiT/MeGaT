%% This is file 'abstract.tex'
%% It is included by hhuthesis-example.tex for hhuthesis.
%%
%% Copyright(C) 2020-2021, Wenhan Cao
%% College of Water Conservancy and Hydropower Engineering, Hohai University.
%%
%% Version:v2.0.0
%% Last update: April 7th, 2021.
%%
%% Home Page of the Project: https://github.com/caowenhan/thesis
%%
%% This file may be distributed and / or modified under the conditions of the
%% LaTeX Project Public License, either version 1.3c of this license or (at your
%% option) any later version. The latest version of this license is in:
%%
%% http://www.latex-project.org/lppl.txt
%%
%% and version 1.3c or later is part of all distributions of LaTeX version
%% 2008/05/04 or later.
%%

\begin{abstract}
	\linespread{1.5}
	作为电磁波谱中能量最高的辐射形式,伽马射线承载着极端天体物理过程的关键信息,是揭示宇宙物质组成、恒星演化机制及宇宙线起源的核心观测窗口。然
	而在伽马射线天文观测实验中,长期存在着所谓的“MeV能量间隙”,即 约在 0.1 到 100 MeV区间内,其间现有实验设备观测灵敏度比其相邻能区相差 1 - 2 个
	数量级,导致正负电子湮灭线(511 keV)、放射性核素衰变等关键物理过程的探测长期受限。
	然而针对MeV能段的高精度测量受限于技术发展而停滞不前。其中伽马射线的高能量和角度分辨率、康普顿散射电子的径迹测量对抑制背景干扰和
	提高灵敏度起到非常关键的作用,是其中的难点,也是新一代MeV伽马射线望远镜研究的重点。
	
	为突破这一技术瓶颈,MeV伽马射线望远镜(MeGaT)创新性地融合了两种尖端探测技术:采用高气压时间投影室(TPC)与像素化碲锌镉(CZT)探测器的复合结构。
	TPC模块通过微网气体探测器(Micromegas)实现康普顿散射电子的高分辨三维径迹重建和损耗能量的测量,可精确反推入射伽马射线的初始方向;
	像素化CZT探测器则凭借优异的高能光子吸收效率与位置灵敏特性,在1-10 MeV能段实现优于1\%的能量分辨率。二者协同工作形成级联探测:TPC捕获康普顿散射
	电子并完成初次作用点定位,CZT探测器记录再沉积光子的精确能量与空间信息,通过双重测量约束大幅降低本底噪声干扰,提高观测的灵敏度。

	% 该实验结合了Micromegas的高气压TPC(时间投影室)和像素读出CZT探测器的优势。TPC能够实现康普顿散射电子
	% 的高分辨3D径迹和能量测量,这对于精确追踪和分析伽马射线的路径和能量分布至关重要。而像素CZT探测器则擅长测量高能量的伽马射线能量和位置,
	% 其高灵敏度和快速响应特性使得它成为高能伽马射线测量的理想选择。 通过将这两种探测技术结合,MeGaT实验能够实现出色的角分辨率和背景抑制。
	% 角分辨率的提高意味着我们能够更精确地定位伽马射线的来源,而背景抑制的增强则有助于减少干扰信号,从而提高观测的灵敏度。\par
	% 本文主要研究内容如下:
\begin{enumerate}
	\item[(1)] MeV伽马天文物理背景及对装置指标需求;
	\item[(2)] MeGaT实验方案与指标预期;
	\item[(3)] MeGaT TPC方案性能模拟优化;
	\item[(4)] TPC探测器构建与性能研究;
	\item[(5)] 数据与模拟对比分析。
\end{enumerate}


\keywords{伽马天文;粒子探测;高能物理;探测技术;MeV伽马射线望远镜}
\end{abstract}

\begin{enabstract}
	\linespread{1.5}
	
	As the highest energy band in the electromagnetic spectrum, gamma rays carry a wealth of information, which is an important way to understand the composition of the universe, the evolution of stars and the origin of cosmic rays.
	In gamma-ray astronomical observation, there has been a ‘MeV energy gap’, which ranges from 0.1 to 100 MeV, and its sensitivity is 1-2 orders of magnitude lower than that of its neighbouring energy regions.
	However, high-precision measurements in the MeV energy band have been stagnant due to technological development, and the high energy and angular resolution of gamma rays and Compton scattering electrons are crucial for suppressing background interference and improving the sensitivity.
	improve the sensitivity play a very crucial role, which is one of the difficulties and the focus of the research on the new generation of MeV gamma-ray telescopes. Against this background, we propose an innovative
	MeV Gamma-ray Telescope (MeGaT) experiment. The experiment combines the advantages of Micromegas' high-pressure TPC (Time Projection Chamber) and the pixel-readout CZT detector, which enables high-resolution 3D traces of Compton scattered electrons.
	The TPC enables high-resolution 3D trajectory and energy measurements of Compton scattered electrons, which are essential for accurately tracing and analysing the path and energy distribution of gamma rays. Pixel CZT detectors, on the other hand, excel in measuring the energy and position of high-energy gamma rays.
	Its high sensitivity and fast response characteristics make it ideal for high-energy gamma-ray measurements. By combining these two detection techniques, the MeGaT experiment is able to achieve excellent angular resolution and background suppression.
	The improved angular resolution means that we are able to locate the source of the gamma rays more accurately, while the enhanced background suppression helps to reduce the interfering signals, thus improving the sensitivity of the observations.\par
	The main research contents of this paper are as follows:


\begin{enumerate}
\item[(1)] MeV gamma astrophysical background and device index requirements;
\item[(2)] MeGaT experimental scheme and index expectations;
\item[(3)] MeGaT TPC scheme performance simulation optimization;	
\item[(4)] TPC detector construction and performance research;
\item[(5)] data and simulation comparative analysis.
\end{enumerate}  
 
\enkeywords{Gamma-ray astronomy; particle detection; high-energy physics; detection technology; MeV Gamma-ray telescopes}

\end{enabstract}
